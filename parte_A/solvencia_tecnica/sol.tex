%
% Carátula oficial de 75.02 Algoritmos y Programación I, cátedra Cardozo.
%
% Basado en el template realizado por Diego Essaya, disponible en
%                                                         http://lug.fi.uba.ar
% Modificado por Michel Peterson.
% Modificado por Sebastián Santisi.

%
% Acá se define el tamaño de letra principal:
%
\documentclass[12pt]{article}

%
% Título y autor(es):
%
\title{Trabajo Práctico N\b o X}
\author{Apellido1\\Apellido2}

%------------------------- Carga de paquetes ---------------------------
%
% Si no necesitás algún paquete, comentalo.
%

%
% Definición del tamaño de página y los márgenes:
%
%\usepackage[a4paper,headheight=16pt,scale={0.7,0.8},hoffset=0.5cm]{geometry}
\usepackage[letterpaper,headheight=16pt,scale={0.7,0.8},hoffset=0.5cm]{geometry}
\usepackage{epstopdf}

%
% Vamos a escribir en castellano:
%
\usepackage[spanish]{babel}
\usepackage[latin1]{inputenc}


%
% Si preferís el tipo de letra Helvetica (Arial), descomentá las siguientes
% dos lineas (las fórmulas seguirán estando en Times):
%
%\usepackage{helvet}
%\renewcommand\familydefault{\sfdefault}

%
% El paquete amsmath agrega algunas funcionalidades extra a las fórmulas. 
% Además defino la numeración de las tablas y figuras al estilo "Figura 2.3", 
% en lugar de "Figura 7". (Por lo tanto, aunque no uses fórmulas, si querés
% este tipo de numeración dejá el paquete amsmath descomentado).
%
\usepackage{amsmath}
\numberwithin{equation}{section}
\numberwithin{figure}{section}
\numberwithin{table}{section}

%
% Para tener cabecera y pie de página con un estilo personalizado:
%
\usepackage{fancyhdr}

%
% Para poner el texto "Figura X" en negrita:
% (Si no tenés el paquete 'caption2', probá con 'caption').
%
\usepackage[hang,bf]{caption2}

%
% Para poder usar subfiguras: (al estilo Figura 2.3(b) )
%
%\usepackage{subfigure}

%
% Para poder agregar notas al pie en tablas:
%
%\usepackage{threeparttable}

%------------------------------ graphicx ----------------------------------
%
% Para incluir imágenes, el siguiente código carga el paquete graphicx 
% según se esté generando un archivo dvi o un pdf (con pdflatex). 

% Para generar dvi, descomentá la linea siguiente:
%\usepackage[dvips]{graphicx}

% Para generar pdf, descomentá las dos lineas seguientes:
\usepackage[pdftex]{graphicx}
\pdfcompresslevel=9

%
% Todas las imágenes están en el directorio tp-img:
%
\newcommand{\imgdir}{includes}
\graphicspath{{\imgdir/}}
%
%------------------------------ graphicx ----------------------------------

% Necesitas este paquete si haces los diagrámas de flujo en el prográma Dia 
%\usepackage{tikz}


%------------------------- Inicio del documento ---------------------------

\begin{document}

%
% Hago que en la cabecera de página se muestre a la derecha la sección,
% y en el pie, en número de página a la derecha:
%
\pagestyle{fancy}
\renewcommand{\sectionmark}[1]{\markboth{}{\thesection\ \ #1}}
\lhead{}
\chead{}
\rhead{\rightmark}
\lfoot{}
\cfoot{}
\rfoot{\thepage}

%
% Carátula:
%
%
% Hago que las páginas se comiencen a contar a partir de aquí:
%
\setcounter{page}{1}

%
% Pongo el índice en una página aparte:
%
%\tableofcontents
%\newpage

%
% Inicio del TP:
%

\section{Solvencia T\'ecnica}

\subsection{Equipo T\'ecnico}

\begin{tabular}{| l | c |}
\hline
Area T\'ecnica & Cant. Personas   \\
\hline
Desarrollo de software & 3 \\
\hline
Investigaci\'on de industria & 3\\
\hline
Implantaci\'on de Software & 2 \\
\hline
Servicios de soporte & 3 \\
\hline
Investigaci\'on de tecnolog\'ia & 3\\
\hline
Aseguramiento de calidad de software & 3\\
\hline
Documentaci\'on de software & 3\\
\hline
\end{tabular} \\
%(participacion de los socios en porcentajes) esta dividido entre los 5 integrantes

\subsection{Conocimiento t\'ecnol\'ogico}
\begin{itemize}
\item Gestor de base de datos relacionales
\begin{itemize}
\item MySQL
\end{itemize} 
\end{itemize}
\begin{itemize}
\item Lenguaje de programaci\'on 
\begin{itemize}
\item PHP
\end{itemize} 
\end{itemize}

\begin{itemize}
\item Herramientas de apoyo para el desarrollo del software
\begin{itemize}
\item Sistema de control de versiones
\item IDE 
\item Lista de correos 
\item Modeladores de bases de datos
\end{itemize} 

\begin{itemize}
\item Experiancia del grupo-empresa
\begin{itemize}
\item Ver en los anexos
\end{itemize} 
\end{itemize}

\end{itemize}



%\begin{figure}[b]
%\centering
%\includegraphics[scale=0.2]{figura1}
%\caption{}
%\label{graf}
%\end{figure}


\end{document}

