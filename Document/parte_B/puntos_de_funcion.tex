\documentclass[12pt]{article}
\usepackage[spanish]{babel}
\usepackage[latin1]{inputenc}

\begin{document}

\section{Estimacion de esfuerzo}

\subsection{Estimacion usando Puntos de Funci�n (PF)}
  Es una m�trica que permite traducir en un n�mero el tama�o de la funcionalidad que brinda un producto de software desde el punto de vista del usuario, a trav�s de una suma ponderada de las caracter�sticas del producto.

\subsubsection{Calculo de puntos de funcion sin ajustar}
EI : Procesos en los que se introducen datos y que suponen la actualizaci�n de cualquier archivo interno. \\
EO: Procesos en los que se env�a datos al exterior de la aplicaci�n. \\
EQ: Procesos consistentes en la combinaci�n de una entrada y una salida, en el que la entrada no produce ning�n cambio en ning�n archivo y la salida no contiene informaci�n derivada. \\
ILF: Grupos de datos relacionados entre s� internos al sistema. \\
EIF: Grupos de datos que se mantienen externamente.\\

\begin{tabular}{|l|c|c|c|r|}
\hline
\textbf{Componente} & \textbf{Bajo} & \textbf{Medio} & \textbf{Alto} & \textbf{Total} \\
\hline
EI  & $4 * 3$ & $0 * 4 $ & $0 * 6$ & $12$ \\
\hline
EO  & $3 * 4$ & $0 * 5$ & $0 * 7$ & $12$ \\
\hline
EQ  & $2 * 3$ & $0 * 5$ & $7 * 6$ & $48$ \\
\hline
ILF & $0 * 7$ & $4 * 10$ & $0 * 15$ & $40$ \\
\hline
EIF & $0 * 5$ & $0 * 7$ & $0 * 10$ & $0$ \\
\hline
\textbf{Total} &&&& 112 \\
\hline
\end{tabular}

\subsubsection{Puntos de funcion ajustados}

\begin{tabular}{|c|c|c|}
\hline
\textbf{Nro de factor} & \textbf{Descripcion} & \textbf{Valor} \\
\hline
1 & Comunicaci�n de Datos & 1 \\
\hline
2 & Proceso Distribuido & 0 \\
\hline
3 & Objetivos de Rendimiento & 1 \\
\hline
4 & Configuraci�n de Explotaci�n Compartida & 0 \\
\hline
5 & Tasa de transacciones & 1 \\
\hline
6 & Entrada de Datos en L�nea & 3 \\
\hline
7 & Eficiencia con el Usuario Final & 2 \\
\hline
8 & Actualizaciones en L�nea & 3 \\
\hline
9 & L�gica de Proceso Interno Compleja & 0 \\
\hline
10 & Reusabilidad del C�digo & 4 \\
\hline
11 & Conversi�n e Instalaci�n contempladas & 1 \\
\hline
12 & Facilidad de Operaci�n & 1 \\
\hline
13 & Instalaciones M�ltiples & 1 \\
\hline
14 & Facilidad de Cambios & 2 \\
\hline
ACT & & 20 \\
\hline
\end{tabular} 

%ACT Ajuste de complejidad tecnica

\subsubsection{Calculo de los puntos de funcion ajustados}

$ACT = 20$

$PFSA = 112$

$PFA = PFSA * [0,65 + [0,01 * ACT]]$

$PFA = 112 * [0,65 + [0,01 * 20]]$

$PFA = 95,2$

\subsubsection{Calculo del esfuerzo}

PFA = 95,2

Horas por Punto de Funcion = 7 \\

\begin{tabular}{|l|l|}
\hline
\textbf{Esfuerzo} & PFA / (1 / Horas por PF) \\
                  & 95,2 / (1 / 7) \\ 
                  & = 666,4 [horas/persona] \\
\hline
\end{tabular}

\subsubsection{Calculo de la duracion del proyecto}

Numero de personas en el proyecto = 3 personas

\begin{tabular}{|l|l|}
\hline
\textbf{Duraci�n} & Esfuerzo / (numero de personas en el proyecto) \\
         & (666,4 [horas/persona]) / (3 [personas]) \\
         & = 222,13 [horas] por miembro \\
\hline
\end{tabular}\\

Las horas de trabajo son de 20,45 horas a la semana por persona.\\

\begin{tabular}{|l|l|}
\hline
\textbf{Horas al mes por trabajador} & 20,45 [horas/semana] * 4 [semanas/mes] \\
                         & = 81,8 [horas/mes] \\
\hline
\end{tabular} \\

\begin{tabular}{|l|l|}
\hline
\textbf{Duraci�n en meses} & (Duraci�n en horas) / (Horas al mes) \\
                           & (222,13 [horas]) / (81,8 [horas/mes]) \\
                           & = 2,71 [meses] \\
\hline
\end{tabular}\\

\subsection{Costo total}

Horas totales a trabajarse en el proyecto calculados
\begin{center}
$222,13$ [horas]
\end{center} 

Suma de costos del personal y los costos operativos:
\begin{center}
$33060 + 1690 = 34750$ Bs.
\end{center}

En el siguiente calculo se detalla el costo por hora:
\begin{center}
$((34750) / 22$ horas laborales $) / 8$ horas al d�a = $197.44$ Bs./Hora
\end{center}

Calculo del costo total:
\begin{center}
$197,44$ Bs./Hora $ * 222,13$ Horas $= 43857,34$ Bs.
\end{center}

\begin{center}
  \begin{tabular}{|l|l|}
    \hline
    \textbf{Costo} & \textbf{Importe (Bs.)}\\
    \hline
    Costo total & 43857,34 \\
    \hline
    Precio de venta incluyendo utilidades (30\%) & 13157,20 \\
    \hline
    \textbf{Precio final} & \textbf{57014.54} \\
    \hline
  \end{tabular} \\
\end{center} 

El precio final ser\'ia de: Cincuenta y Siete Mil Catorce 54/100 Bs.



\end{document}
