\documentclass{article}
\usepackage[latin1]{inputenc}
\usepackage{wallpaper}
%\usepackage[letterpaper,headheight=16pt,scale={0.7,0.8},hoffset=0.5cm]{geometry}
\usepackage[letterpaper]{geometry}
\usepackage[spanish]{babel}

\usepackage{amsmath}
\numberwithin{equation}{section}
\numberwithin{figure}{section}
\numberwithin{table}{section}

\usepackage[hang,bf]{caption}
\pagestyle{empty}

%\usepackage[pdftex]{graphicx}
%\pdfcompresslevel=9

\newcommand{\imgdir}{includes}
\graphicspath{{\imgdir/}}

\topmargin 0in

\ULCornerWallPaper{1}{membrete_bw.pdf}

\begin{document}


\section{PROPUESTA DE SERVICIOS}

\subsection{PROPOSITO}
La presente propuesta t�cnica tiene como prop�sito exponer nuestra alternativa de soluci�n al problema de la empresa de TIS para una juez virtual para la Olimpiada Internacional de Informatica. Adem�s el documento servir� de gu�a para el proceso de desarrollo del producto de software y formar� parte indivisible del contrato estipulado entre la empresa TIS y el proponente, esperando que cumpla y satisfaga con las exigencias emitidas por parte de la empresa licitante.

\subsection{OBJETIVOS}

\subsubsection{OBJETIVO GENERAL}

Desarrollar un juez virtual, que funcione a nivel departamental, que permita a olimpistas enfrentarse en un modo de evaluacion, como el que se usa la competencia mundial.

\subsubsection{OBJETIVO ESPEC�FICOS}
\begin{itemize}
    \item Desarrollo de componentes de gestion de usuarios. %anadir, modificar, quitar usuarios, generar reportes de usuarios
    \item Desarrollo de componentes de gestion de evaluacion. %
    \item Desarrollo de componentes de evaluador de problemas.% recibe solucion, corre casos de prueba, y contabiliza el resultados
    \item Desarrollo de componentes del modulo de competicion.% lleva a cabo una competicion, controla el tiempo, registra puntajes de los equipos
    \item Desarrollo de componentes del modulo de practicas % Este permite a un equipo o olimpista a practicar localmente, sin necesidad de estar en una competicion, es parecido al modulo de competicion. (pueden heredar funcionalidades de una clase padre)

\end{itemize}

\subsubsection{TECNOLOGIA A UTILIZAR}
Para el cumplimiento de los objetivos de esta propuesta y las condiciones
establecidas en el pliego de especificaciones de la empresa TIS, se ha
considerado desarrollar el sistema web de administraci�n de cursos y notas,
con las siguientes herramientas de software:

\begin{center}
    \begin{tabular}{|l|l|c|}
    \hline
    \textbf{Herramienta}    & \textbf{Software}  \\
    \hline
    Servidor web                        & Apache web server   \\
    Gestor de base de datos             & MySQL               \\
    Lenguaje de programaci�n            & PHP                 \\
    Librerias de apoyo                  & Symphony            \\
                                        & jQuery              \\
    Sistema de Control de Versiones     & Git                 \\
    Entorno integrado de desarrollo     & Netbeans            \\
    \hline
    \end{tabular}
\end{center}

\subsection{Tipos de usuarios}
\begin{enumerate}
    \item Olimpista.
    \item Administrador.
    \item Miembro de comite academico.
\end{enumerate}

\subsection{Funcionalidades globales}
\begin{description}
    \item[Gesti�n de usuarios]
    Componente encargado de la gesti�n de usuarios...
    \item[Gesti�n de evaluacion]
    Componente que recibe una solucion a un problema y la evalua, luego devuelve un puntaje equivalente.
    \item[Gesti�n de equipos] %hay equipos en las olimpiadas??
    Componente encargado de la gestion de equipos de trabajo entre los
    usuarios del sistema.
    \item[]
    \item[]
    \item[Reporte de ranking de olimpiadas]
    %\item[Seleccion del tipo de participacion que puede ser competicion o practica]
    \item[Modo de practica en la olimpiada]
    \item[Modo de competencia en la olimpiada]
    \item[Gestion de problemas]
    Permite la subida problemas, modificacion, dar de baja.
    \item[Gestion de competicion]
    Crear comp (anadir olimpistas), iniciar competicion, finalizar, generacion de reportes
    % Gestion de problemas
    % Gestion de evaluacion
    % Reportes: Ranking, Estadisticas de problemas
    % 
\end{description}

\subsubsection{MODULOS DEL SISTEMA}
Una vez desarrollado el sub-sistema administrador de modulos, esta
herramienta ser� provista de los modulos necesarios para gestionar
los siguientes elementos:
\begin{itemize}
    \item Ficheros.
    \item Avisos, actividaddes y eventos.
    \item Generaci�n dinamica de contenido.
    \item Foros.
    \item Valoraciones de usuarios.
    \item Comentarios.
    \item Busquedas.
    \item Etiquetas dinamicas.
    \item Evaluaciones y calificaciones.
\end{itemize}

\section{Plan de pagos}

El plan de pagos se detalla a continuacion, sin tomar en cuenta la tasa de interes:

% 600$ * 3 = 1800 * 3 meses = 5400

% firma 900$ 
% 1er hito 1800$
% 2do hito 1800$
% 3er hito 1800$

% tomar en cuenta que no se trabajan 8 hrs. al dia
% 15 semanas de trabajo en total
% 6+5+5+2.5+6+5 = 29.5 horas a la semana
% calculando sueldo sale 4$/hora
% 4$ * 29.5 horas * 15 semanas * 3 personas = 5310$ total
% 5310 * 1.20 de cuestiones adm = 6372$;

%costos fijos = sueldos y salarios
%costos var. = servicios basicos (luz, agua, internet, telefono), insumos (papaleria, agua botelln, suminitros oficina)
%material de trabajo(limpieza)

%activos fijos mensual
%gerente gral  1000$
%secretaria     600$ 
%limpieza       200$
%administracion 600$
%contabilidad   400$
%ingenieros     700$ * 3 
%-------------------
%total         4900$

%costos variables (Bs.)
%luz                 300
%agua                150
%internet           1000  
%telefono            100
%suministros ofic.   100
%agua botellon        40
%-----------------------
%total              2890

%total costos: 2890 + 4900 * 6.96 = 36994.0 Bs.
%por semana = (36994.0 / 22) / 8 = 210.19 bs se paga por hora

%horas que se van a trabajar en el proyecto:
%29.5 h * 15 sem = 442.5 horas

%442.5 * 210.19 = 93009.075 bs ~ 13363$
%


%Nro pago & Fase & Iteracion & Monto & Detalle & Entregable

%1 & - & - & 

\section{Propuesta economica}

\subsection{Costos operativos}
Estos costos se refieren a los servicios basicos y otros suministros, un calculo aproximado en bolivianos por mes ser\'ia:

\begin{tabular}{|l|l|}
\hline
Luz & 300 \\
\hline
Agua & 150 \\ 
\hline
Internet & 1000 \\
\hline
Telefono & 100 \\ 
\hline
Suministros de oficina & 100 \\
\hline
Agua embotellada & 40 \\
\hline
\textbf{Total} & 1690 \\
\hline
\end{tabular}

\subsection{Costos del personal}
Estos costos se refieren al salario del personal de la empresa, un calculo aproximado en bolivinos por mes ser\'ia:\\

\begin{tabular}{|l|l|}
\hline
Gerente general & 6960 \\
\hline
Secretaria & 2784 \\
\hline
Limpieza & 1044 \\ 
\hline
Administracion & 4176 \\
\hline
Contabilidad & 3480 \\
\hline
Ingenieros (3) & 14616 \\
\hline
\textbf{Total} & 33060 \\
\hline
\end{tabular}

\subsection{Costo total}
Nuestro grupo-empresa realizara el desarrollo del sistema por 29.5 horas a la semana segun nuestro calendario definido en la parte A. Tambien se toma en cuenta que el proyecto esta programado para realizarse durante 15 semanas.

Horas totales a trabajarse en el proyecto:
\begin{displaymath}
29.5 * 15 = 442.5 Horas
\end{displaymath} 

Suma de costos del personal y los costos operativos:
\begin{displaymath}
33060 + 1690 = 34750 Bs.
\end{displaymath}

En el siguiente calculo se detalla el costo por hora:
\begin{displaymath}
((34750) / 22 horas laborales ) / 8 horas al dia = 197.44 Bs./Hora
\end{displaymath}

Costo total del producto:
\begin{displaymath}
197.44 Bs./Hora * 442.5 Horas = 87367.2 Bs.
\end{displaymath}

\begin{tabular}{|l|l|}
\hline
\textbf{Costo} & \textbf{Importe (Bs.)}\\
\hline
Costo total & 87367.2 \\
\hline
Precio de venta incluyendo utilidades (20\%) & 17473.44 \\
\hline
\textbf{Precio final} & \textbf{104840.64} \\
\hline
\end{tabular} \\

El precio final ser\'ia de Cien mil cuatro ochocientos cuarenta 64/100 Bs 

\end{document}
