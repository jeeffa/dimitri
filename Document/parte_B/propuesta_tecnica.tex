\documentclass{article}
\usepackage[latin1]{inputenc}
\usepackage{wallpaper}
%\usepackage[letterpaper,headheight=16pt,scale={0.7,0.8},hoffset=0.5cm]{geometry}
\usepackage[letterpaper]{geometry}
\usepackage[spanish]{babel}

\usepackage{amsmath}
\numberwithin{equation}{section}
\numberwithin{figure}{section}
\numberwithin{table}{section}

\usepackage[hang,bf]{caption}
\pagestyle{empty}

%\usepackage[pdftex]{graphicx}
%\pdfcompresslevel=9

\newcommand{\imgdir}{includes}
\graphicspath{{\imgdir/}}

\topmargin 0in

\ULCornerWallPaper{1}{membrete_bw.pdf}

\begin{document}


\section{Propuesta de servicios}

\subsection{Prop�sito}

La presente propuesta t�cnica tiene como prop�sito exponer nuestra alternativa de soluci�n al problema de la empresa de TIS para una juez virtual para la Olimpiada Internacional de Inform�tica. Adem�s el documento servir� de gu�a para el proceso de desarrollo del producto de software y formar� parte indivisible del contrato estipulado entre la empresa TIS y el proponente, esperando que cumpla y satisfaga con las exigencias emitidas por parte de la empresa licitante.

\subsection{Objetivo}

\subsubsection{Objetivo general}

Desarrollar un juez virtual, que funcione a nivel departamental, que permita a olimpistas enfrentarse en un modo de evaluaci�n, como el que se usa la competencia mundial.

\subsubsection{Objetivos espec�ficos}
\begin{itemize}
    \item Desarrollo de componentes de gesti�n de usuarios. %anadir, modificar, quitar usuarios, generar reportes de usuarios
    \item Desarrollo de componentes del modulo de evaluador. %
%    \item Desarrollo de componentes de evaluador de problemas.% recibe solucion, corre casos de prueba, y contabiliza el resultados
    \item Desarrollo de componentes del gesti�n de competici�n.% lleva a cabo una competicion, controla el tiempo, registra puntajes de los equipos
    \item Desarrollo de componentes del gesti�n de problemas 
    %\item Integracion de todos los componentes
    \item Realizacion de la gesti�n de calidad del sistema integrado
% Practicas - Este permite a un equipo o olimpista a practicar localmente, sin necesidad de estar en una competicion, es parecido al modulo de competicion. (pueden heredar funcionalidades de una clase padre)

\end{itemize}

\subsubsection{Tecnolog�a a utilizar}
Para el cumplimiento de los objetivos de esta propuesta y las condiciones establecidas en el pliego de especificaciones de la empresa TIS, se ha considerado desarrollar el sistema del \guillemotleft Juez Virtual\guillemotright con las siguientes herramientas de software:

\begin{center}
    \begin{tabular}{|l|l|}
    \hline
    \textbf{Herramienta}    & \textbf{Software}  \\
    \hline
    Servidor web                        & Apache              \\
    Gestor de base de datos             & MySQL               \\
    Lenguaje de programaci�n            & PHP                 \\
    Framework de apoyo                  & Symfony             \\
    Biblioteca de apoyo                 & jQuery              \\
    Sistema de control de versiones     & Git                 \\
    Entorno integrado de desarrollo     & Netbeans            \\
    Composici\'on de documentaci\'on    & \LaTeX              \\
    Editor de texto plano               & vim                 \\
    \hline
    \end{tabular}
\end{center} 

Las versiones a ser utilizadas de Apache, MySQL y PHP seran definidas de acuerdo al software instalado en el laboratorio de TIS.


\subsection{Tipos de usuarios} \label{users}
\begin{enumerate}
    \item Olimpista.
    \item Administrador.
    \item Miembro de comit� acad�mico.
\end{enumerate}

\subsection{Funcionalidades globales}
\begin{description}
    \item[Gesti�n de usuarios]
    La gestion de usuarios consta de la creacion, modificacion y depuracion de usuarios, los tipos de usuarios se especifican en  \ref{users}. Cada tipo de usuario tendra establecido sus privilegios con respecto al Sistema.
    \item[Modulo evaluador]
    Consta de un sub-sistema que recibe una soluci�n a un problema y ejecuta los casos de prueba correspondientes al problema, luego devuelve un puntaje de acuerdo a los resultados de la ejecucion de los casos de prueba. 
    \item[Gesti�n de competici�n] 
    Permite crear una competicion entre olimpistas, se podra seleccionar el modo de la competicion, es decir si la competicion va ser verdadera o solo sera una practica. Una vez creada la competicion se la podra dar inicio y tambien se la podra finalizar. Tambien se contempla la creacion de reportes relacionados a la competicion, como ser un ranking de una competicion determinada.
    \item[Gesti�n problemas]
    Permite la subida nuevos problemas, la modificaci�n de problemas y dar de baja un problema.
    %\item[Seleccion del tipo de participacion que puede ser competicion o practica]
%    \item[Modo de practica en la olimpiada]
%    \item[Modo de competencia en la olimpiada]
%    \item[Gesti�n de problemas]
    % Gestion de problemas
    % Gestion de evaluacion
    % Reportes: Ranking, Estadisticas de problemas
    % 
\end{description}

\subsubsection{Modulos del sistema}
Una vez desarrollado el sub-sistema administrador de m�dulos, esta
herramienta ser� provista de los m�dulos necesarios para gestionar
los siguientes elementos:
\begin{itemize}
    \item Ficheros.
    \item Avisos, actividades y eventos.
    \item Generaci�n din�mica de contenido.
    \item Foros.
    \item Valoraciones de usuarios.
    \item Comentarios.
    \item B�squedas.
    \item Etiquetas din�micas.
    \item Evaluaciones y calificaciones.
\end{itemize}

\section{Plazo de conclusi\'on del contrato}
Una vez entregado el software y firmado un documento de conformidad por parte de la empresa TIS, se dar\'a por concluida la relaci\'on contractual con la empresa TIS. Tales acciones en ning\'un caso podr\'an exceder el d\'ia 6 de diciembre del 2013.

\section{Propuesta econ�mica}

\subsection{Costos operativos}
Estos costos se refieren a los servicios b�sicos y otros suministros, un calculo aproximado en bolivianos por mes ser\'ia:\\

\begin{center}
  \begin{tabular}{|l|r|}
  \hline
  Luz & 300 \\
  \hline
  Agua & 150 \\ 
  \hline
  Internet & 1000 \\
  \hline
  Tel�fono & 100 \\ 
  \hline
  Suministros de oficina & 100 \\
  \hline
  Agua embotellada & 40 \\
  \hline
  \textbf{Total} & 1690 \\
  \hline
  \end{tabular}
\end{center}


\subsection{Costos del personal}
Estos costos se refieren al salario del personal de la empresa, un calculo aproximado en bolivianos por mes ser\'ia:\\

\begin{center}
  \begin{tabular}{|l|r|}
  \hline
  Gerente general & 6960 \\
  \hline
  Secretaria & 2784 \\
  \hline
  Limpieza & 1044 \\ 
  \hline
  Administraci�n & 4176 \\
  \hline
  Contabilidad & 3480 \\
  \hline
  Ingenieros (3) & 14616 \\
  \hline
  \textbf{Total} & 33060 \\
  \hline
  \end{tabular}
\end{center}

\subsection{Costo total}
Nuestro grupo-empresa realizara el desarrollo del sistema por $20.45$ horas a la semana seg�n nuestro calendario definido en la parte A. Tambi�n se toma en cuenta que el proyecto esta programado para realizarse durante 17 semanas, incluyendo la realizaci�n de la propuesta.

Horas totales a trabajarse en el proyecto:
\begin{center}
$20.45 * 17 = 347.65$ Horas
\end{center} 

Suma de costos del personal y los costos operativos:
\begin{center}
$33060 + 1690 = 34750$ Bs.
\end{center}

En el siguiente calculo se detalla el costo por hora:
\begin{center}
$((34750) / 22$ horas laborales $) / 8$ horas al d�a = $197.44$ Bs./Hora
\end{center}

Calculo del costo total:
\begin{center}
$197.44$ Bs./Hora $ * 347.65$ Horas $= 68641.12$ Bs.
\end{center}

\begin{center}
  \begin{tabular}{|l|l|}
    \hline
    \textbf{Costo} & \textbf{Importe (Bs.)}\\
    \hline
    Costo total & 68641,12 \\
    \hline
    Precio de venta incluyendo utilidades (20\%) & 13728,22 \\
    \hline
    \textbf{Precio final} & \textbf{82369,34} \\
    \hline
  \end{tabular} \\
\end{center} 

El precio final ser\'ia de: Ochenta y Dos Mil Trescientos Sesenta y Nueve 34/100 Bs 

\section{Plan de pagos}
Los pagos se realizar\'an en la finalizaci�n de cada hito.\\

\begin{center}
  \begin{tabular}{|l|l| r @{,} l |}
    \hline
    \textbf{Pago} & \textbf{Evento} & \multicolumn{2}{c|}{\textbf{Monto (Bs.)}}  \\
    \hline
    \hline
    1 & 1"er hito, 9 de septiembre & 8075&3 \\
    \hline
    2 & 2"er hito, 6 de diciembre  & 52489&42 \\
    \hline
    3 & 3"er hito, 20 de diciembre & 8076&73 \\
    \hline
  \end{tabular}
\end{center}

\end{document}
